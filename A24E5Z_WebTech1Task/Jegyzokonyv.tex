\documentclass[12pt,a4paper]{article}

\usepackage[magyar]{babel}
\usepackage[utf8]{inputenc}
\usepackage[T1]{fontenc}
\usepackage{newtxtext,newtxmath}

\usepackage{setspace}
\onehalfspacing
\frenchspacing
\usepackage{geometry}
\geometry{margin=3cm}
\usepackage{ragged2e}
\justifying
\usepackage[hidelinks]{hyperref}
\usepackage{graphicx}
\usepackage{float}
\usepackage{listings}


\lstset{
  basicstyle=\ttfamily\small,      
  backgroundcolor=\color{gray!10},
  frame=single,
  breaklines=true,
  showstringspaces=false,
  numbers=left,
  numberstyle=\tiny\color{gray},
  literate=
    {á}{{\'a}}1
    {é}{{\'e}}1
    {í}{{\'\i}}1
    {ó}{{\'o}}1
    {ö}{{\"o}}1
    {ő}{{\H{o}}}1
    {ú}{{\'u}}1
    {ü}{{\"u}}1
    {ű}{{\H{u}}}1
}

\lstdefinelanguage{json}{
    basicstyle=\ttfamily\small,
    showstringspaces=false,
    breaklines=true,
    frame=single,
    backgroundcolor=\color{gray!5},
    stringstyle=\color{red!70!black},
    keywordstyle=\color{blue},
    keywords={true,false,null},
    morestring=[b]",
    literate={á}{{\'a}}1 {é}{{\'e}}1 {í}{{\'i}}1 {ó}{{\'o}}1 {ö}{{\"o}}1
             {ő}{{\H{o}}}1 {ú}{{\'u}}1 {ü}{{\"u}}1 {ű}{{\H{u}}}1
             {Á}{{\'A}}1 {É}{{\'E}}1 {Í}{{\'I}}1 {Ó}{{\'O}}1 {Ö}{{\"O}}1
             {Ő}{{\H{O}}}1 {Ú}{{\'U}}1 {Ü}{{\"U}}1 {Ű}{{\H{U}}}1
             {„}{{"}}1 {”}{{"}}1 {“}{{"}}1
}


\begin{document}

\begin{titlepage}
    \centering
    \vspace*{3cm}
    {\Huge\bfseries JEGYZŐKÖNYV \par}
    \vspace{1.5cm}
    {\Large Web Technológiák 1. \par}
    \vspace{0.5cm}
    {\Large Féléves feladat \par}
    \vspace{0.5cm}
    {\Large Wild West Saloon weboldal \par}

    \vfill
    \begin{flushright}
        \textbf{Készítette:} Tóth Péter\\[0.3cm]
        \textbf{Neptunkód:} A24E5Z\\[0.3cm]
        \textbf{Dátum:} 2025. november
    \end{flushright}

    \vspace{2cm}
    {\large Miskolc, 2025}
\end{titlepage}

% ---------- TARTALOMJEGYZÉK ----------
\tableofcontents
\newpage

\section{Bevezetés}
A web technológiák napjainkra a digitális kommunikáció és információcsere alapvető pilléreivé váltak. A HTML, CSS, JavaScript és a hozzájuk kapcsolódó könyvtárak, mint a jQuery, olyan szabványos eszközkészletet biztosítanak, amelyek segítségével egységesen, minden eszközön elérhető módon jeleníthető meg tartalom.

\section{Feladat leírása}
A jelen dokumentum a Wild West Saloon vadnyugattal foglalkozó többoldalas weboldal felépítését, működését és technikai megoldásait mutatja be. A weboldal HTML, CSS, JavaScript, jQuery és JSON technológiákra épül.

\section{Weblap felépítése}

\subsection{Navigációs gombok}

Az oldal tetején találhatóak navigációs gombok. A jobb oldali részben találhatóak a html dokumentumok közötti navigáláshoz szükséges gombok ezek a "Kezdőlap", "Legendák", "Fegyverek", "Média" és "Űrlap" gombok, ha rájuk viszed az egeret besárgulnak ezzel jelezve a felhasználónak, hogy kattinthatóak.

\begin{figure}[H]
\centering
\includegraphics[width=\textwidth]{kepek/PDFhez/navigacio.jpg}
\caption{A navigáció sáv}
\label{fig: navigacios gombok}
\end{figure}

\subsection{Oldalsáv}

Az oldalsávon lévő linkek a könnyebb navigálásért felelősek. Minden fejezet egy külön linkkel rendelkezik, ezért azonnal ahhoz tudunk ugrani, amire szükségünk van.

\begin{figure}[H]
\centering
\includegraphics[width=\textwidth]{kepek/PDFhez/oldalsav.jpg}
\caption{Az oldalsáv}
\label{fig: oldalsav gombok}
\end{figure}

\subsection{Kép megjelenítő}

Egy kép jelenik meg ha a kurzurt a történetnél lévő névre visszük rá ami oldalt fog megjelenni. A név színe átvált fehérre, jelezve a felhasználónak, hogy ott valami történik.

\begin{figure}[H]
\centering
\includegraphics[width=\textwidth]{kepek/PDFhez/kepmegjelenito.jpg}
\caption{Kép megjelenítő}
\label{fig: kep megjelenito}
\end{figure}

\subsection{Videó kezelő gombok}

Minden videó alatt találhatóak gombok, amelyek a videó alapvető kezelésére szolgál. Az első sorban a videó elindítására, leállítására, illetve elindít/leállít gombok találhatóak. A második sorban a hangszabályzó gombok, az első  némítja a videót lejjebb vesz a hangerőt, a második lejjebb vesz a hangerőt, a harmadik felveszi a maximum hangerőre. A harmadik sorban a videó lejátszási sebessége állítható be. Az öt gomb sorrendben a következő sebességértékeket kínálja: 0.25x, 0.5x, 1x (normál sebesség), 1.5x és 2x. A kiválasztott sebességre kattintva a videó azonnal az adott tempóban folytatja a lejátszást.

\begin{figure}[H]
\centering
\includegraphics[width=\textwidth]{kepek/PDFhez/video_kezelo_gombok.jpg}
\caption{Videó kezelő gombok}
\label{fig: video kezelo gombok}
\end{figure}

\section{Scriptek}

\subsection{Kép megjelenítő}

Ez a JavaScript-kód kizárólag 1024\,px-nél szélesebb képernyőkön (tehát elsősorban asztali környezetben) fut. Amikor a felhasználó a kurzorral egy szereplő neve fölé viszi az egeret (azok az elemek, amelyek a \texttt{.legend-name} osztályt viselik), a szkript megkeresi a hozzá tartozó kártyaelem (\texttt{.legend-card}) legközelebbi szülőjét, majd onnan kinyeri a szereplő képének elérési útját. Ezt a képet azonnal betölti egy előre definiált előnézeti konténerbe (az azonosítója \texttt{legend-hover-preview}). Eközben az előnézeti elem megkapja az \texttt{active} osztályt, amelynek köszönhetően láthatóvá válik és szép fade-in animációval megjelenik (általában jobb oldalon, fix pozícióban).
Amint az egér elhagyja a nevet, az \texttt{active} osztály eltávolításra kerül, így az előnézet elhalványul és eltűnik. Egy további eseményfigyelő a teljes dokumentum \texttt{mouseleave} eseményét is kezeli: ha az egér egyenesen kilép a böngészőablakból, akkor is garantáltan eltűnik az előnézet, elkerülve a „beragadt” állapotot.

\begin{lstlisting}[language=java, caption={Kép megjelenítés}]
if (window.innerWidth > 1024) {
    const preview = document.getElementById('legend-hover-preview');
    const previewImg = document.getElementById('hover-preview-img');

    document.querySelectorAll('.legend-name').forEach(name => {
        name.addEventListener('mouseenter', function() {
            const card = this.closest('.legend-card');
            const imgSrc = card.querySelector('.legend-img').src;

            previewImg.src = imgSrc;
            preview.classList.add('active');
        });

        name.addEventListener('mouseleave', function() {
            preview.classList.remove('active');
        });
    });

    document.addEventListener('mouseleave', function(e) {
        if (!e.relatedTarget && !e.toElement) {
            preview.classList.remove('active');
        }
    });
}
\end{lstlisting}


\subsection{Animáció}

képek és bizonyos tartalmi elemek megjelenését látványos, görgetéshez kötött animáció teszi dinamikussá. Ezt a CSS \texttt{@keyframes appear} szabály és a modern \texttt{scroll-driven animations} (görgetésvezérelt animációk) kombinációja biztosítja:

\begin{lstlisting}[language=java, caption={Animáció}]
@keyframes appear {
    from{
        opacity: 0;
        scale: 0.5
    }
    to {
        opacity: 1;
        scale: 1;
    }
}

animation: appear linear;
animation-timeline: view();
animation-range: entry 0% cover 40%;
    
\end{lstlisting}


\subsection{Adat-lekérdezés}

Ez a program a DomRead-el szemben csak szimplán kiírja az adatokat, hanem emberileg olvasható és feldolgozható formában és mennyiségben írja ki azt.

\begin{lstlisting}[language=java, caption={Javaban rendezett adatkiírás}]
            System.out.println("3. Minden tankolás dátuma:");
            NodeList tankList = doc.getElementsByTagName("Tankolas");
            for (int i = 0; i < tankList.getLength(); i++) {
                Element t = (Element) tankList.item(i);
                String datum = t.getElementsByTagName("Datum").item(0).getTextContent();
                System.out.println("  - " + datum);
            }
            System.out.println();
\end{lstlisting}

Ebben a kódrészletben először kinyer a program minden hasznos elementet és attribútumot, majd azt egy kompakt módon olvashatóan kiírja.

\subsection{Json-ből beolvasás}

A TWWguns.html oldalon a táblázatok adatai dinamikusan, egy külső JSON fájlból kerülnek betöltésre. A program minden egyes fegyver objektumot beolvas, majd annak tulajdonságait egy előre elkészített HTML-es táblázat struktúrába illeszti, ezzel automatikusan felépítve a teljes táblázatot. Így az adatok karbantartása egyszerűen a JSON fájl szerkesztésével megoldható, nincs szükség a HTML kód manuális módosítására.

\begin{lstlisting}[language=java, caption={Json beolvasás}]
fetch("TWWguns.json")
            .then(res => res.json())
            .then(data => loadWeapons(data));

        function loadWeapons(weapons) {

            // Oldalsó menü generálása
            const menu = document.getElementById("weapon-menu");
            menu.innerHTML = weapons.map(w =>
                `<li><a href="#${w.id}">${w.title}</a></li>`
            ).join("");

            // Fegyver blokkok ide kerülnek
            const container = document.getElementById("weapon-container");

            weapons.forEach(w => {

                // Táblázat sorok
                const rows = Object.entries(w.data)
                    .map(([key, val]) => `<tr><th>${key}</th><td>${val}</td></tr>`)
                    .join("");

                container.innerHTML += `
                    <section id="${w.id}" class="weapon-block">
                        <div class="weapon-img">
                            <img src="${w.img}" alt="${w.title}">
                        </div>
                        <div class="weapon-info">
                            <h2 class="weapon-title">${w.title}</h2>
                            <table>${rows}</table>
                        </div>
                    </section>
                `;
            });
        }
\end{lstlisting}

\begin{lstlisting}[language=json, caption={Egy fegyver adatait tároló JSON objektum (részlet)}]
{
    "id": "colt-saa",
    "title": "Colt Single Action Army „Peacemaker”",
    "img": "kepek/Colt.png",
    "data": {
      "Típus": "Forgópisztoly (single-action revolver)",
      "Űrméret": ".45 Colt",
      "Tárkapacitás": "6 lőszer",
      "Gyártás": "1873-napjainkig",
      "Becenév": "„The Peacemaker” - „Az egyenlítő”"
    }
}
\end{lstlisting}


\end{document}